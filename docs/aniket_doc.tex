\documentclass[12pt,a4paper]{scrartcl}

\usepackage[top=1.5in]{geometry}
\usepackage{amsmath}
\usepackage{hyperref}
\hypersetup{
  colorlinks, linkcolor=brown
}

\title{\textbf{Gymkhana Nominations Portal}}
\subtitle{Summer Project - Programming Club}
\date{27 May 2017}
\author{Aniket Pandey}

\begin{document}
\maketitle

\section{Acknowledgement}
\textit{I would like to thank Programming Club for giving me the opportunity to create the portal and learn about backend web development in Django}.

\section{Repository}
The project repository can be found \textbf{\href{https://github.com/SummerCamp17/Gymkhana-Nominations}{here}} and the live project is hosted \textbf{\href{https://gymkhana.pythonanywhere.com}{here}}.

\subsection{Mentors}
\begin{itemize}
	\item Kunal Kapila
	\item Yash Srivastav
	\item Pratham Verma
\end{itemize}


\section{Description}
The main aim of the project is to develop a portal through which admins can release nominations for their child posts. The target group can apply for the same. The portal can keep track of each and every process the application goes through, e.g release, submission, approval from higher authority, interview remarks, final selection and increment in power if selected. The detailed documentation can be found \textbf{\href{http://aniketpandey.com/gymkhana-doc}{here}}. 

\section{Timeline}
\subsection{Week 1}
\begin{enumerate}
	\item We started off by creating a simple web-app about a library through which normal users can issue a book and depending on the availability, librarian (aka admin) can either reject or accept the issuance. This was meant as an exercise for the actual project, and to get familiar with django.
	\item Learnt about custom backends and integrated IIT K user authentication by overriding the default authentication. So that users can login through their \textsc{cc} id. However, we were not able to fetch their profile data, e.g Name, Roll No, etc.
	\item Decided to implement Model Forms and Generic Views in the code. This decision led to easy editing and adding of user profile (fixed \#2).
	\item Extended user model by importing current user and adding few required fields. Final code till now looks clean and user friendly.
\end{enumerate}

\subsection{Week 2}
\begin{enumerate}
	\item After creating the basic template, we implemented \textit{Dynamic Forms} as a separate app. It is possible to add various kind of questions in the form. After the form was finished, we integrated it into the nominations app. 
	\item Created the topmost(independent) Club Model. Interlinked Users, Clubs and Posts. Now user can create multiple post nominations within a particular club. 
	\item Created a filter for User Profile so that while releasing a nomination, it is possible to specify the target group (e.g users who are supposed to apply for the post). It remains to combine Forms and Filters for a nomination request.
\end{enumerate}

\subsection{Week 3}
\begin{enumerate}
	\item I created a separate view for all the Clubs, in which it is possible to have a look at the current post holders in all the clubs.
	Although the clubs don't have the typical heirarchial structure, they are just used to differentiate between the posts at the same level.
	\item We finally integrated the Filters with Nomination form. Using that, it is possible to specify for whom this nomination is being released. On completion of Backend, we discussed the possiblity of using \textit{Angular 2} as a front end framework, however, due to some compatibility issue of Angular with Dynamic Forms, we had to use Bootstrap for writing frontend.
	\item Finished off the nomination heirarchy, now all the nominations that are released are first sent to the parent post, then on his approval , the process continues till it reaches the Gen-Sec. On his approval, the nominations are released to the public for applying.
	Meanwhile, the nomination creator and all the approving authority can view the applicants' answers.
	\item Created the responsive template for the Index, Profile and the Nomination-Detail view. Replaced the original with the new template. 
\end{enumerate}

\subsection{Week 4}
\begin{enumerate}
	\item Completed the frontend templates, replaced all the previous templates with the new one. Bootstrapped all the forms to make it look attractive. 
	\item Fixed the error in which the new user was not able to see his index view after logging in. Passed the username before the user logged-in, so as to create a user instance and allowing him to update his profile.
	\item Added commenting functionality in the nominee's filled form. So during the interview, panel can record comments about the candidate.
	\item Finished off the main structure of the project, only thing that remains to be done is to enhance the look by altering the font-sizes, background colour etc.
	\item Started writing an API-reference of the code for future maintenance. 
\end{enumerate}

\subsection{Week 5}
\begin{enumerate}
	\item In this week, I worked on inculcating ratification process of the posts in our portal. For that, we created a separate post Senate which is the parent of every Gen-Sec. Gen-Sec can send final list to senate after whose approval, posts are assigned to the applicants.
	\item We also worked on the feature that a particular post-holder can cancel approval requests by which the editing rights are a brought back to the concerned post form the parent post. Right now, only that level which has the approval can edit anything regarding the nomination results.
	\item We also modified commenting in the applicant's response. It is now related to commenting feature in facebook. That is, you write something, it gets published beneath the user response, and everyone can view it. 
\end{enumerate}

\subsection{Week 6}
\begin{enumerate}
	\item We started working on a new feature of our app, the Student Search. This search feature is different from the conventional student search where you can search for a person only through his Club/Post. For example you want to know who all are in SnT Council. Or who is the Secretary of Films Club etc.
	\item Designed search layout. Made sure that search feature was equally responsive on mobile devices. Fixed few bugs about absence of post-holders. In that case, returned empty list. Also created a public profile, by clicking on an individual search result. You can view the concerned individual's Current Post. Post Histories and his basic profile.
	\item Wrote a separate view for senate approvals. Made sure that senate did not appear in 'My Posts' dropdown. and if senate was the only post that user had, removed the Post dropdown form the panel. 
\end{enumerate}

\subsection{Week 7}
\begin{enumerate}
	\item With our Project finally completed, we started testing our code. Worked on few examples, and removed few creepy bugs.
	\item Made a feature to add separate interview panelists. Customised it so that the post holders of parent post are already assigned the panelist tag. And re-adding is removed. 
	\item Corrected few bugs on returning multiple values and Object not being present in database.
\end{enumerate}

\end{document}
